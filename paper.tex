%%
%% If you intend to use figures of formats jpg, png or pdf and want the
%% output to be immediately a pdf file, compile with pdflatex.
%%
%% If you want to use eps or ps figures, your output will be a dvi
%% file that can be converted to ps and pdf formats. In this case you
%% should compile your document with latex.
%%
%%This template is compatible with both methods.
%%




\documentclass[a4paper]{IEEEtran}
\usepackage{pdfpages}
\usepackage[utf8]{inputenc}
\usepackage[T1]{fontenc}
\usepackage{graphicx}
\usepackage{amsmath}
\usepackage{amsfonts}
\usepackage{caption}
\usepackage{subcaption}
\usepackage[caption=false, ...]{subfig}

\title{{ \normalsize Master Thesis} \\
	Bluetooth Low Energy Supported Indoor Location}
\author{
	Ricardo Martins {\tt ricardo.pachecomartins@gmail.com}
	Instituto Superior T\'{e}cnico}
\date{\today}

\begin{document}
\maketitle

\begin{abstract}

	Abstract

\end{abstract}

\section{Introduction}
\label{sec:Introduction}

Indoor positioning systems have greatly evolved in the past few years due to the great success of its counter-part, the global positioning system (GPS), in outdoor environments but failure to reproduce the same results in indoor environments. Since GPS is an outdoor position systems and is based on a network of satelite, when the required scenario for position tracking is inside a building, new constraints are presented onto the process such as the attenuation and reflection of eletromagnetic waves upon collision with building walls and obstacles \cite{surveygps}. As such there was a need to find reliable indoor systems that by nature would already be able to heavily reduce the impact of some of the mentioned constraints.

In order to understand indoor position there is a need to understand the full scope of variables that come to surface when moving from outdoor to indoor. When developing a system there is a need to make sure that it can tackle the chalenges such as small space dimension which reforce the need for higher precision, a higher probability of non existent line of sight, influence of obstacles such as walls, furniture, moveable objects such as doors and human beings\cite{reviewtechniques}. All of the previously mentioned affect the way electromagnetic waves propagate in an indoor environment leading to problems related to severe multipath and reflection on existent surfaces \cite{surveywireless}. Besides propagation challenges, there are energy consumption, accuracy and deployment costs that play a critical role in deciding the viability of a proposed indoor location technique.

With the evolution of mobile devices there has been a sizeable number of different technologies that can possibly be used in indoor locationing \cite{surveywireless,survey2,survey1} such as GPS-based technologies, using high sensitivity antenas to overcome GPS's indoor issues, RFID , Wireless LAN and Bluetooth among others, allowing even for hibrid systems which make use of more than one of the technologies mentioned above. 

Another important aspect besides the chosen technology is the location detection technique which are widely varied in terms of accuracy and complexity. The simplest available category would be proximity detection which has Cell of Origin (CoO) as its most famous technique, allowing for room-based detection by assuming that the mobile target is at the cell with the strongest receiving power. This technique is widely used by system using RFID, Bluetooth and Infrared. The next category would be triangulation which uses the geometric properties of the triangle to obtain a position. The method utilized to obtain the values used in the position's calculation can be angle-based, which have implementation costs are their worst limitation; time-based, using Time-of-flight (ToF) or Round Trip Time (RTT) to obtain measures of distance from beacons; signal property based, using received signal strength indicator (RSSI) as a metric, altough it's only possible of using this metric with radio signals; dead reckoning, which calculates the actual position utilizing the last determined position and incrementing based on estimate of the device's speed.

The solution in this paper is based on the Bluetooth Low energy technology using the CoO method to obtain the device's location and as such for what technologies and location methods are concerned, only the used ones will be focused. 

In this paper is structured in a way that in \ref{sec:related} an overview of Bluetooth low energy (BLE) and existent projects and their associated architecture are presented while shedding light on the benefits and limitations of the chosen technologies, in \ref{sec:structure} the proposed system is described based on technology and method used as well as the architecture of the would system, \ref{sec:performance} analyzes the different aspects of the presented solution in terms of energetic efficiency, accuracy and response timings, finalizing in \ref{sec:future} by overviewing what could be carried out in order to further develop the existing work and by concluding the paper.


- Falta de framework, não tenho bases, vale a pena mencionar na introdução?

\section{Related Word}
\label{sec:related}

Bluetooth is a wireless technology that was created in 1994 with the objective of replacing cables connecting fixed or portable devices. At this point in time Bluetooth Special Interest Group is in charge of developing and managing this technology characterized by its robustness, low energy consuption and low cost. The Bluetooth Low Energy protocol was introduced with the Bluetooth Core Specification version 4 (also called Bluetooth Smart) circa 2010 alongside two other protocols.  Out of the three, BLE standed out for its lower power consuption, lower complexity and lower cost, while allowing for  device discovery, connection establishment and connection mechanisms. 

The BLE radio operates at the 2.4GHz band and employs a frequency hopping transceiver to combat interference and fading. It also employs two multiple access schemes: FDMA used to separate the 40 available physical channels, 37 of them are used as data channels and the remaining as advertising channels and TDMA in a polling scheme that is used when one device transmits a packet at a predetermined time and a corresponding device responds with a packet after a predetermined interval.

 \begin{figure}[htp]
	\centering
		\includegraphics[width=0.5\linewidth]{figures/BLEArchitecture.png}
	\caption[Bluetooth Low Energy Architecture]{Bluetooth low energy architecture}
	\label{fig:BLEarchitecture}
\end{figure}

When looking at the BLE's architecture , which can be seen at \ref{fig:BLEarchitecture}, the Generic Access Profile (GAP) is one of the most relevant since it's responsible for working in conjunction with Generic Attribute (GATT) to define the base funcionality of BLE devices. The services that are made available by the GAP are: BLE device discovery, connection modes, security, authentication, association models and service discovery.
In addiction to this, it also defines four different roles to describe a device, allowing for the controllers to be optimized in funtion of the device's desired roles: 
Broadcaster, role optimized for transmitter-only applications; 
Observer, role optimized for receiver-only applications and it's complementary to the broadcaster role;
Peripheral, role optimized for devices that only want to suppot a single connection, allowing for a much less complex controller due to the fact that it only needs to support the slave role and not the master one; 
Central, role supports multiple connections and funtions as the initiator for all of them. These connection are all made with Peripheral devices and its controller must support the master role in a connection and allow for more complex funtions, in comparison to the remaining roles.

Another important component is the Attribute Protocol (ATT) Protocol which is responsible for implementing the Peer-to-peer(P2P) protocol between an attribute server and client. This communication happens in a dedicated fixed  channel and a server can send through it responses, notifications and indications, while the client can send requests, commands and confirmations. The ATT allows the clients to read and write values of attributes on a peer device acting as a server.

The last component is the Generic Attribute (GATT) Profile which is responsible for creating a framework for the ATT, in which it's represented the funcionalities of an ATT server. This profile describes the hierarchy of services, characteristics and attributes existent in the server and provides an interface for discovering, reading, writing and indicating service characteristics and profiles.

Bluetooth low energy devices utilize profiles which define the required functionalities of the device. It also defines application behaviour and data formats and as such when two devices comply with all the requirements of a Bluetooth profile, application interoperability is achieved. Each Bluetooth profiles describes its requirements necessary for devices to create a connection, to find available services and connection information required for making application level connections.

\begin{figure}
	\centering
		\includegraphics[width=0.5\linewidth]{figures/profile.png}
	\caption[Gatt-based profile hierarchy]{Gatt-based profile hierarchy}
	\label{fig:profile}
\end{figure}

The base profile that any Bluetooth system needs to include is the GAP. From this point, any additional profile implemented will be a superset of GAP, where GATT is included and specifies the profile hierarchy, or the structure in which profile data is exchanged. Figure  ~\ref{fig:profile} shows the hierarchy in a Gatt-based profile, with the profile being the top level and services and characteristics below. 
A profile is composed by one or more services. A service is a collection of data and associated behaviors to accomplish a particular function or feature of a device or portions of a device. A service is composed of characteristics and/or references to other services.
A Characteristic is a value that is used in a service that has properties and configuration information that descrive how the value should be accessed as well as information on how to display the value. A characteristic is defined by its declaration, its properties, its value and may also be defined by its descriptor, which describes the value or permit configuration of the server relative to the value.


When looking at what's possible to achieve using the BLE technology there is the example of Apple's creation iBeacon which was presented in 2013 with the porpuse of implementing proximity sensing systems. The device is capable of playing on the broadcaster role and as such its objective is to send nearby compatible receivers certain information. Some examples of application are to track customers or trigger location-based actions on devices such as push notifications or checking in on social media, with pratical cases such as the usage of iBeacons by McDonalds to offer special offers to their customers in their fast-food stores. An indoor location system utilizing this technology was presented by Jingjing Yang et al \cite{ibeacon}, where these devices were used to indicate a pacient of his whereabouts through the proximity sensing proprieties and this information was later transfered over to a server in order to give clients a variaty of different services, from pacient counting, to nearby department's information and offer indoor guidance to the nearest available bed. 

When utilizing BLE for indoor location the usual metric used to calculate distances is the RSSI. This metric withint the context of bluetooth brings to surface several issues such as the fact that RSSI as a metric is very accurate only when the target is within a meter of the beacon, since the value decreases as the inverse of the square of the distance to the beacon . As such when developing solutions for indoor location that require system with high accuracy capable of tracking moving objects, the usage of RSSI can't be utilized without further work. Faragher et al \cite{bleacc} tackled one of the techniques used to improve BLE system's accuracy, fingerprinting, by verifying the effects caused by the device deployment density within the required location. This experiment also puts into evidence one of the downsides of the bluetooth technology being that its scalability is low, besides requiring higher density in order to increase accuracy, due to their low range any need to increase coverage leads to increased costs.


Cricket
Active Bat
Radar
active badge

recent stuff

In 2011, LifeMap \cite{lifemap} introduced one of the first to attempt to track indoor location with unconstrained phone placement. Not Bluetooth

Zonith \cite{zonith} introduced a bluetooth based location system with the objective of tracking the position of workers in dangerous environments. Any device registered in the zonith implemented network would be continuously tracked and accounted for in each of the system's functionalities such as, sounding an alarm whenever a lone worker doesn't move or responde within a time interval (Lone worker protection) or providing a quick an precise location of any worker that has requested for help. This system's installation requires planing of the best locations to place the beacons and number required of beacons in order to be able to provide enough courage and make sure the system provides the required quality.

lifemap
zonith


Chen et al. [2010] present an inquiry-based locating approach using Bluetooth
RSS measurements. There also exist several commercial Bluetooth-based indoor positioning
systems, such as ZONITH [Teldio 2006], which is used to assist people who
work in hazardous environments.


- BLE Projects
	

- Architecture similar stuff?

\section{Implementation}
\label{sec:struture}


The solution presented in this paper was made with the objective of presenting a generic indoor location system using bluetooth low energy. The system's architecture is presented in figure ~\ref{fig:architecture} and is divided three parts: the device, the application and the server. For each of these parts an explanation will be given, containing a description of each of its components specific to the presented system alongside the requirements for each to work.

\begin{figure}
	\centering
		\includegraphics[width=0.5\linewidth]{figures/profile.png}
	\caption[Gatt-based profile hierarchy]{Gatt-based profile hierarchy}
	\label{fig:architecture}
\end{figure}

\subsection{ BLE beacon}
\label{subsec:beacon}

\subsection{ Server}
\label{subsec:beacon}

\subsection{ Application}
\label{subsec:beacon}



- Overview of Architecture

- Location Provider
	- BLE Search
	- GATT Comunication
	- Server Calculation
	- Voting System  and CoO

- Google Maps
	- Indoor maps 
	- Map insertion possibility

Images:
	- Architecture Overview
	- Location Provider Diagram
	- App Screen



\section{Performance Analysis}
\label{sec:performance}
 



\section{Future Work}
\label{sec:future}





\begin{thebibliography}{9}

\bibitem{surveygps}
 \texttt{Hakan Koyuncu, Shuang Hua Yang ;A Survey of Indoor Positioning and Object Locating Systems http://paper.ijcsns.org/07_book/201005/20100518.pdf}

\bibitem{surveywireless}
 \texttt{Hui Liu; Survey of Wireless Indoor Positioning Techniques and Systems http://www.pitt.edu/~dtipper/2011/Survey1.pdf} 

\bibitem{reviewtechniques}
 \texttt{Zahid Farid, Rosdiadee Nordin, and Mahamod Ismail; Recent Advances in Wireless Indoor Localization Techniques and System}

\bibitem{survey2}
 \texttt{Hakan Koyuncu, Shuang Hua Yang; A Survey of Indoor Positioning and Object Locating Systems; http://paper.ijcsns.org/07_book/201005/20100518.pdf}

\bibitem{survey1}
 \texttt{Jiang Xiao, Zimu Zhou,Youwen Yi and Lionel M. Ni; A Survey on Wireless Indoor Localization from the Device Perspective; http://www.tik.ee.ethz.ch/~zzhou/paper/csur16-xiao.pdf}

\bibitem{blecore}
\texttt{ }

\bibitem{ibeacon}
\texttt{ Jingjing Yang, Zhihui Wang and Xiao Zhang ;An iBeacon-based Indoor Positioning Systems for Hospitals; http://www.sersc.org/journals/IJSH/vol9_no7_2015/16.pdf}

\bibitem{bleacc}
\texttt{ R. Faragher and R. Harle; An Analysis of the Accuracy of Bluetooth Low Energy for Indoor Positioning Applications; http://www.cl.cam.ac.uk/~rmf25/papers/BLE.pdf}

\bibitem{lifemap}
\texttt{Yohan Chon and Hojung Cha; LifeMap: A SmartphoneBased Context Provider for Location-Based Services; https://pdfs.semanticscholar.org/cb50/e7344494f8b8fdfeba160f814cbfe3ce3356.pdf}

\bibitem{zonith}
\texttt{http://media.teldio.com/pdfs/Teldio_Indoor%20Positioning%20White%20Paper_v2.1a.pdf}

 
 
\end{thebibliography}

\end{document}
